%
% Projekt: proj2.tex
% Autor:   Michal Ľaš
% Datum:   05.03.2023
% 


\documentclass[twocolumn, 11pt, a4paper]{article}

\usepackage[czech]{babel}
\usepackage{times}
\usepackage[utf8]{inputenc}
\usepackage[IL2]{fontenc}
\usepackage[left=1.4cm,text={18.2cm, 25.2cm},top=2.3cm]{geometry}
\usepackage{amsthm, amssymb, amsmath}


\newtheorem{definition}{Definice}
\newtheorem{sentence}{Věta}


\begin{document}

\begin{titlepage}
    \begin{center}
            \textsc{\Huge Vysoké učení technické v~Brně \\}
            \vspace{0.5em}
            \textsc{\huge Fakulta informačních technologií \\}
        \vspace{\stretch{0.382}}
            {\LARGE Typografie a~publikování\,--\,2. projekt \\ 
            \vspace{0.4em}
            Sazba dokumentů a~matematických výrazů}
        \vspace{\stretch{0.618}}
    \end{center}
    {\Large 2023 \hfill Michal Ľaš (xlasmi00)}
\end{titlepage}


\section*{Úvod}

V~této úloze si vyzkoušíme sazbu titulní strany, matematických vzorců, prostředí a~dalších textových struktur 
obvyklých pro technicky zaměřené texty\,--\,například Definice~\ref{def1} nebo rovnice~\eqref{eq:3} na straně~\pageref{eq:3}. Pro vytvoření těchto odkazů používáme 
kombinace příkazů \verb|\label|, \verb|\ref|, \verb|\eqref| a~\verb|\pageref|. Před odkazy patří nezlomitelná mezera. Pro zvýrazňování textu jsou zde 
několikrát použity příkazy \verb|\verb|~a~\verb|\emph|.

Na titulní straně je použito prostředí \verb|titlepage| a~sázení nadpisu podle 
optického středu s~využitím \emph{přesného} zlatého řezu. Tento postup byl probírán na přednášce. 
Dále jsou na titulní straně použity čtyři různé velikosti písma a~mezi dvojicemi řádků textu je použito 
odřádkování se zadanou relativní velikostí 0,5\,em a~0,4\,em\footnote{Nezapomeňte použít správný typ mezery mezi číslem a~jednotkou.}.

\section{Matematický text}

V~této sekci se podíváme na sázení matematických symbolů a~výrazů v~plynulém textu pomocí prostředí \verb|math|. 
Definice a~věty sázíme pomocí příkazu \verb|\newtheorem| s~využitím balíku \verb|amsthm|. Někdy je vhodné použít konstrukci 
\verb|${}$| nebo \verb|\mbox{}|, která říká, že (matematický) text nemá být zalomen.

\begin{definition}
    \label{def1}
    \emph{Zásobníkový automat (ZA)} je definován jako sedmice tvaru \mbox{$ A = (Q, \Sigma, \Gamma, \delta, q_0, Z_0, F) $}, kde:
    \begin{itemize}
        \item $ Q $ je konečná množina \emph{vnitřních (řídicích) stavů,}
        \item $ \Sigma $ je konečná \emph{vstupní abeceda,} 
        \item $ \Gamma $ je konečná \emph{zásobníková abeceda,}
        \item $ \delta \text{ je \emph{přechodová funkce} } Q \times (\Sigma \cup \{\epsilon\}) \times \Gamma \rightarrow 2^{Q \times \Gamma^*}$,
        \item $ q_0 \in Q $ je \emph{počáteční stav,} $ Z_0 \in \Gamma $ je \emph{startovací symbol zásobníku a}~$ F \subseteq Q $ je množina \emph{koncových stavů.} 
    \end{itemize}

\end{definition}
 
Nechť \mbox{$\ P\ = (Q, \Sigma, \Gamma, \delta, q_0, Z_0, F) $} je ZA. \emph{Konfigurací} nazveme trojici $ (q, w, \alpha) \in Q \times \Sigma^* \times \Gamma^* $, kde $ q $ je aktuální stav vnitřního řízení, 
$ w $ je dosud nezpracovaná část vstupního řetězce a~\mbox{$ \alpha = Z_{i_1} Z_{i_2} \ldots Z_{i_k} $} je obsah zásobníku.

\subsection{Podsekce obsahující definici a~větu}

\begin{definition}
    \label{def2}
    \emph{Řetězec $ w $ nad abecedou $ \Sigma $ je přijat ZA $ A $} jestliže \mbox{$ (q_0, w, Z_0) \underset{A}{\overset{*}{\vdash}} (q_F, \epsilon, \gamma) \text{ pro nějaké } \gamma \in \Gamma^* \text{ a } q_F \in F $}.
    Množina \mbox{$ L(A) = \{ w \mid w \text{ je přijat ZA } A \} \subseteq \Sigma^* $} je \emph{jazyk přijímaný ZA} $ A $. 
\end{definition}

\begin{sentence}
    \label{sent1}
    Třída jazyků, které jsou přijímány ZA, odpovídá \emph{bezkontextovým jazykům}.
\end{sentence}

\section{Rovnice}

Složitější matematické formulace sázíme mimo plynulý text pomocí prostředí \verb|displaymath|. 
Lze umístit i~několik výrazů na jeden řádek, ale pak je třeba tyto vhodně oddělit, například příkazem \verb|\quad|.
\begin{displaymath}
    1^{2^3} \neq \Delta _{\Delta _{\Delta^3}^2}^1
    \quad
    y_{22}^{11} - \sqrt[9]{x + \sqrt[7]{y}} 
    \quad
    x > y_1 \leq y^2
\end{displaymath}
V rovnici~\eqref{eq:2} jsou využity tři typy závorek s různou \emph{explicitně} definovanou velikostí. Také nepřehlédněte, že nasledující 
tři rovnice mají zarovnaná rovnítka, a~použijte k~tomuto účelu vhodné prostředí.
\begin{eqnarray}
    -\cos^2\,\beta &=& \frac{\frac{\frac{1}{x} + \frac{1}{3}}{y} + 1000}{\prod\limits _{j=2}^8 q_j} \label{eq:1}\\
    \biggl(\Bigl\{ b \star \bigl[3 \div 4\bigr] \circ a~\Bigr\}^\frac{2}{3}\biggr) &=& \log_{10}\,x \label{eq:2}\\
    \int_a^b f(x)\,\mathrm{d}x &=& \int_c^d f(y)\,\mathrm{d}y \label{eq:3}
\end{eqnarray}
V~této větě vidíme, jak vypadá implicitní vysázení limity \mbox{$ \lim_{m \to \infty}\,f(m) $} v~normálním odstavci textu. 
Podobně je to i~s~dalšími symboly jako \mbox{$ \bigcup_{N \in \mathcal{M}} N $ či $ \sum _{i=1}^m x_i^2 $}. 
S~vynucením méně úsporné sazby příkazem \verb|\limits| budou vzorce vysázeny v podobě \mbox{$ \lim\limits _{m \to \infty} f(m) \text{ a} \sum\limits _{i=1}^m x_i^4 $}.

\section{Matice}

Pro sázení matic se velmi často používá prostředí \verb|array| a~závorky (\verb|\left|, \verb|\right|).
$$
\mathbf{B} = 
\left|
    \begin{array}{cccc}
        b_{11} & b_{12} & \dots & b_{1n} \\
        b_{21} & b_{22} & \dots & b_{2n} \\
        \vdots & \vdots & \ddots & \vdots \\
        b_{m1} & b_{m2} & \dots & b_{mn}
    \end{array}
\right|
=
\left|
    \begin{array}[]{cc}
        t & u \\
        v & w 
    \end{array}
\right|
=
tw - uv
$$

$$
\mathbb{X} = \mathbf{Y} \Longleftrightarrow
\left[
    \begin{array}[pos]{ccc}
        & \Omega + \Delta & \hat{\psi} \\
        \vec \pi & \omega & 
    \end{array}
\right]
\neq 42
$$

Prostředí \verb|array| lze úspěšně využít i jinde, například na pravé straně následující rovnice. 
Kombinační číslo na levé straně vysázejte pomocí příkazu \verb|\binom|.

\begin{displaymath}
    \binom{n}{k} = 
    \left\{
        \begin{array}{c l}
            0 & \text{pro } k < 0 \\
            \frac{n!}{k!(n-k)!} & \text{pro } 0 \leq k \leq n \\
            0 & \text{pro } k > 0
        \end{array} 
    \right.
\end{displaymath}

\end{document}
